In this section, we introduce the two publicly accessible datasets we will use for our experiments, ETH~\cite{ETH-biwi} and UCY~\cite{UCY-crowds}. They are widely used datasets that contain real-world human trajectories. Researchers at ETH and UCY took videos at different locations and manually annotated them with time steps of 0.4 seconds. The datasets of ETH contain 2 subsets: eth (with 365 pedestrians) and hotel (with 420 pedestrians). The UCY datasets contain 3  subsets: zara01 (with 148 pedestrians), zara02 (with 204 pedestrians) and univ (with 850 pedestrians). The UCY datasets contain 3 subsets: zara01 (with 148 pedestrians), zara02 (with 204 pedestrians) and univ (with 850 pedestrians). All these pedestrian trajectories were recorded in 4 different scenarios with different maps, obstacles and times. People have various goals, so there are different directions and velocities.

\subsection{Data acquisition}
The original datasets can be downloaded directly from the official download links (ETH - BIWI Walking Pedestrians dataset~\footnote{\url{https://data.vision.ee.ethz.ch/cvl/aem/ewap_dataset_full.tgz}},  UCY - Crowded Data ~\footnote{\url{https://graphics.cs.ucy.ac.cy/research/downloads/crowd-data}}).


\subsection{Data preprocessing}
In our experiments, we use the data parsed by SocialWays directly. In our experiments, we use the data parsed by Social Ways directly. For the ETH dataset, the parsing process can be illustrated by the Social Ways project, and for the UCY dataset, the parsing process is currently not included in the project and is not publicly available. But the parsed data is provided by Social Ways and is available for public download~\footnote{\url{http://www.dropbox.com/sh/lh1s41d1pqp8cbx/AAD4sB1JAiZIkCL7LHht-S4Ca}}.