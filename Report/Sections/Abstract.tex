Pedestrians trajectory prediction has received significant importance in recent years due to the increase in autonomous systems, specifically in the automotive industry because of the rising number of self-driving vehicles. Different methods can be used to accomplish the trajectory prediction task, but deterministic and non-deterministic methods are the most commonly used methods. In this report, we used the non-deterministic method, for trajectory prediction and tried to overcome this problem by using sequence prediction and InfoGAN (an information-theoretic extension of Generative Adversarial Network): our recurrent sequence-to-sequence model observes the small subset of pedestrian trajectory as history and predicts the future trajectory, to predict it uses a pooling mechanism to aggregate the information across the other nearby trajectories in the dataset. Our InfoGAN implementation for trajectory prediction learns and predicts disentangled representations of plausible future trajectories in a completely unsupervised manner. Through experiments on several datasets we demonstrate that our approach outperforms prior work in terms of accuracy, variety, collision avoidance, and computational complexity.