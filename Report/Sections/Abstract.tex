Pedestrians trajectory prediction has received significant importance over the last few years due to the increase in autonomous robots and vehicles, specifically in the automotive industry because of the rising number of self-driving vehicles. Different methods can be used to accomplish the trajectory prediction task, however deterministic and non-deterministic methods are the most commonly used methods. GANs have been used to solve the prediction problem, but we want to find the latent factors that affect the prediction. Finding latent space is usually hard because the latent space is unknown and uncontrollable. In this report, we use the non-deterministic method of socialGAN and introduce  InfoGAN(an information-theoretic extension of GAN) to socialGAN for trajectory prediction and try to disentangle the latent factors. After training the model, we change the latent codes of the generator in GAN and generate trajectories prediction of several samples. Through experiments, we disentangled one code by observing the trajectories images with changing latent codes. The code we found affects the trajectory direction of the prediction.