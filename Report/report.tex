\documentclass[sigconf]{acmart}
\settopmatter{printacmref=false}

%% remove copyright and further ACM refs for the Lab
\setcopyright{none}
\copyrightyear{}
\acmDOI{}
\acmISBN{}
\acmConference[Data Science Lab]{Data Science Lab}{2019}{Uni Passau}

%% end of the preamble, start of the body of the document source.
\begin{document}

%%
%% The "title" command has an optional parameter,
%% allowing the author to define a "short title" to be used in page headers.
\title{A Report of Trajectory Prediction based on Deep Learning}

%% author list and team

\author{Chenxiao Tian}
\email{tian01@ads.uni-passau.de}
\affiliation{\institution{University of Passau}}

\author{firstname lastname}
\email{mail@example.org}
\affiliation{\institution{}}

\author{Yashu Wang}
\email{wang52@ads.uni-passau.de}
\affiliation{\institution{University of Passau}}

\author{Zubair Ahmed}
\email{ahmed08@ads.uni-passau.de}
\affiliation{\institution{}}

\renewcommand{\shortauthors}{team 4: Trajectory Prediction}

% \begingroup
% \renewcommand{\cleardoublepage}{}
% \renewcommand{\clearpage}{}
% \section*{Phases Assignment}
% \begin{itemize}
%     \item \boldsymbol{Phase1} Chenxiao Tian
%     \item \boldsymbol{Phase2} Zubair Ahmed
%     \item \boldsymbol{Phase3} Yashu Wang
%     \item \boldsymbol{Phase4} Rita Akhmetova
% \end{itemize}
% \endgroup





%%
%% The abstract is a short summary of the work to be presented in the
%% article.

\begin{abstract}
% A generally good scheme for an abstract is the following:
% \begin{itemize}
%  \item The problem?
%  \item Our solution.
%  \item Our solution in detail.
%  \item So what?
% \end{itemize}
% (1-2 sentences each).
% You don't need to provide an abstract during the first 3 phases, as you most likely will not have an answer to all items.
% It should be included in your phase 4 submission and the final document.

Pedestrians trajectories prediction has become a popular topic in recent years. Many approaches can be used to accomplish the prediction, including deterministic and non-deterministic ways. In this report we conduct research and discussion about how to use the a non-deterministic method i.e., deep learning to solve the prediction problem. The algorithm we are going to use is the combination of SocialGAN and InfoGAN. First, we train a model by the adopted algorithms. Second, during the research, we propose our research questions and try to answer it in the report.
\end{abstract}

%%
%% This command processes the author and affiliation and title
%% information and builds the first part of the formatted document.
\maketitle
\section*{Phases Assignment}
\begin{itemize}
    \item \boldsymbol{Phase1} Chenxiao Tian
    \item \boldsymbol{Phase2} Zubair Ahmed
    \item \boldsymbol{Phase3} Yashu Wang
    \item \boldsymbol{Phase4} Rita Akhmetova
\end{itemize}
\section*{Introduction}

Systems and electronic devices that use pedestrian trajectories prediction can be found everywhere in our daily life, e.g., service robots, automatic drive and city planning etc. Hence, it is useful and important to make predictions about pedestrians movement. Many researches have proposed miscellaneous approaches that tackle this problem. Helbing and Molnar~\cite{Helbing95} propose the Social Force model. Yi~\cite{Yi15} introduces the factor of stationary group to the modeling of pedestrians trajectories with an energy map. The aforementioned ways are deterministic ways for prediction, they can not utilize the valuable information in the trajectories data.

Over the last few years, following the widespread usage of machine learning and deep learning, researchers use various neural networks to tackle the trajectories prediction problem. Zhou et al.~\cite{Zhou} build a linear dynamic system, applying Expectation Maximization (EM) algorithm to estimate parameters, to learn motion patterns in crowded scenes. Altché~\cite{Altche17} proposes a method that predicts the trajectory on the highway using Long Short-Term Memory (LSTM). Alahi et al.~\cite{Alahi16} give a sequence model based on LSTM as well as a social pooling that aggregates the human-human interaction in a scene.

However, these approaches mentioned previously learn only the pattern of human motion from data. Predicting human trajectory is a complex task. This is because both internal and external stimuli, such as intentions and other directly or indirectly observable influences, can affect human motion, as mentioned in the survey~\cite{humanmotionsurvey}. In addition to the location, which is usually recorded in the dataset, many factors that are not explicitly recorded in the dataset, such as speed, direction, or even not recorded, such as route and human intent. Recent researches have shown that Generative Adversarial Network (GAN) can better capture these uncertainties with latent space and thus naturally preserve multimodality. Gupta et al.~\cite{Gupta_2018_CVPR} used GAN and a Pooling Module to predict socially acceptable trajectories and found that certain directions in the latent space are related to direction and velocity. What is more, The study of Amirian et al.~\cite{Amirian_2019_CVPR_Workshops} has shown that InfoGAN, an information-theoretic extension to the Generative Adversarial Network~\cite{infogan}, partly improves the performance on commonly used datasets that have the largest variance in the prediction distribution, while still leaving some room for improvement.

Even though these researches give various effective models that fulfill the prediction task and attempt to encompass hidden aspects that influence the trajectory, they have not disentangled these factors in the latent space. If we know the factors that affect pedestrians' trajectory and apply these factors in specific scenarios. We can obtain better performance of prediction on various distributed datasets and to mitigate the limitations of the observed data. Therefore, we decide to consider the hidden factors behind different datasets.

In this study, we focus on what factors we can obtain that influence human trajectories and try to develop a model that can be controlled by these factors. we assume that different datasets have different static environments and so the data in a dataset share some specific common features. We consider three factors: obstacles (obstacles information such as the presence of static obstacles and the coordinates), maps (geometry and topology), and semantics (environment semantics such as no-go-zones, crosswalks, sidewalks, or traffic lights) in static environments, which are denoted by the survey~\cite{humanmotionsurvey}. We propose to develop a conditional generation model that is controlled by factor $c$ to have different static environments. We demonstrate that human movement is influenced by these three factors we consider in a static environment, and so with inputting different factors in static environments, our model can achieve better performance on different datasets.

\section{Problem statement}

In this paper, our goal is to develop a controllable generative model to predict pedestrian trajectories.
Consider the problem of predicting the future trajectory of each pedestrian. Let $(x_i^t, y_i^t)$ denote the position of the $i$ pedestrian at time $t$, and a sequence of coordinates  $[(x_i^{t}, y_i^{t}), (x_i^{t+1}, y_i^{t+1}), ..., (x_i^{t+n}, y_i^{t+n})]$ denote the trajectory of pedestrians from time $t$ to $t+n$.

Given the observed trajectory of $n_{obs}$ steps $X_i^t = [(x_i^{t}, y_i^{t}), \\ (x_i^{t+1}, y_i^{t+1}), ..., (x_i^{t+n_{obs}}, y_i^{t+n_{obs}})]$, with certain condition $c$ and random variable $z$,  we want to fit a function to generate the prediction of trajectory for the next $n_{pred}$ steps $Y_i^t = [(x_i^{t+n_{obs}+1}, y_i^{t+n_{obs}+1}), \\ (x_i^{t+n_{obs}+ 2}, y_i^{t+n_{obs}+2}), ...,  (x_i^{t+n_{obs}+ n_{pred}}, y_i^{t+n_{obs}+n_{pred}})]$. That is
$$ Y_i^t = f(X_i^t \vert c, z) $$

The prediction $Y_i^t$ is conditioned on the vector $c$, where consist of $(c_1, c_2, c_3)$. So we can control the factors of obstacles, maps, and semantics respectively. These factors might vary over time.


% Describe your problem statement clearly and short.
% Your algorithm will most likely not bring world-peace, but solve a particular problem.
% Narrow down your problem to be as specific as possible.
% The more specific your problem, the better the chance to find a suitable solution. After a solution is found, the complexity can still be increased.
% For example ``the beer price is too high, we aim to reduce it'' is not a very precise problem statement. What is ``too high''? Which kind of beer? At which area/country? At which occasion (restaurant,supermarket,special event,etc.)? ...?

% Besides the problem statement, you should also indicate your approach on how you plan to solve it and how you plan to evaluate your solution in this section.

% Most likely, the problem you are going to address is not completely new, but at least a similar problem has previously been addressed by others already.
% Therefore, research how others solved this or similar problems, i.e. get familiar with the state of the art and related work.
% You may then apply (and adapt/extend) an existing solution to a similar problem to your problem at hand.

% \section{Data Acquisition \& Pre-Processing}
% Make sure to precisely describe what you did and why.
% The reader of your report should be able to reproduce the steps.
% Therefore, it is for example not sufficient if you write ``we tokenized the sentences''.
% You need to describe how the tokenization was done exactly, i.e. which regular expression or library/method you used with which parameters.

% Frameworks/libraries are mentioned in footnotes instead of in the references section.
% For example ``we used the \emph{TweetTokenizer} from the NLTK\footnote{\url{https://www.nltk.org/api/nltk.tokenize.html}} toolkit with the default parameters to tokenize our tweets''.

% Reasoning is highly important.
% It should be obvious to the reader, why you do something in a particular way.
% The following subsections provide hints on what to include in your report.

% \begin{center}
%  \noindent\fbox{%
%     \parbox{0.3\textwidth}{%
%         (!) required \newline
%         (*) if it applies to your project
%     }%
% }

% \end{center}

% \subsection{Data acquisition}
% \begin{itemize}
%  \item source of data (!)
%  \item means of acquisition (!)
%  \item reasoning (!) (i.e. why you chose that kind of data, that source, why you crawled it in that way, etc.)
% \end{itemize}
% As the datasets are mostly pre-defined (i.e. in most of the topics you do not need to acquire the data yourself), this part can be short.

% \subsection{Data preprocessing}
% \begin{itemize}
%  \item filtering / grouping / labeling (*)
%  \item lemmatization/stemming (*)
%  \item other (*)
%  \item statistics on the data (!) (e.g. volume, classes, distributions, correlations, etc.)
%  \item reasoning (!)
% \end{itemize}

% \subsection{Feature engineering}
% \begin{itemize}
%  \item input x and output y of the system (*)
%  \item feature extraction (*)
%  \item feature transformation (*)
%  \item reasoning (!)
% \end{itemize}



% \section{Model implemenation}
% \subsection{Methodology / Proposed solution / Technique}
% An in-depth description of your solution to the problem. (e.g. input - output, system description, ML techniques, etc.)

% \subsection{Experimental Setup}
% An in-depth description of the experimental design which enables an objective quantification of the quality of your solution. (e.g. dataset, baselines, metrics, etc.).
% Might be moved to the next section (Phase 4).

% \section{Evaluation}
% \subsection{Results}
% The evaluation of your solution by means of the experiments.
% Figures and statistics provide hard facts about the quality of your solution from different viewpoints.

% \subsection{Discussion}
% So what?
% How well could we solve the problem?
% What are the limitations?
% Open ends.


% \section*{Template information (remove this section in your report)}
% \label{sec:template-information}
% Instructions in the following sections are included from the original ACM template sample file.
% The article template's documentation, available at
% \url{https://www.acm.org/publications/proceedings-template}, has a
% complete explanation and tips for effective use.

% \subsection*{Sectioning Commands}

% Your work should use standard \LaTeX\ sectioning commands:
% \verb|section|, \verb|subsection|, \verb|subsubsection|, and
% \verb|paragraph|.

% \subsection*{Tables}

% The ``\verb|acmart|'' document class includes the ``\verb|booktabs|''
% package --- \url{https://ctan.org/pkg/booktabs} --- for preparing
% high-quality tables.

% Table captions are placed {\itshape above} the table.

% Because tables cannot be split across pages, the best placement for
% them is typically the top of the page nearest their initial cite.  To
% ensure this proper ``floating'' placement of tables, use the
% environment \textbf{table} to enclose the table's contents and the
% table caption.  The contents of the table itself must go in the
% \textbf{tabular} environment, to be aligned properly in rows and
% columns, with the desired horizontal and vertical rules.  Again,
% detailed instructions on \textbf{tabular} material are found in the
% \textit{\LaTeX\ User's Guide}.

% Immediately following this sentence is the point at which
% Table~\ref{tab:freq} is included in the input file; compare the
% placement of the table here with the table in the printed output of
% this document.

% \begin{table}
%   \caption{Frequency of Special Characters}
%   \label{tab:freq}
%   \begin{tabular}{ccl}
%     \toprule
%     Non-English or Math&Frequency&Comments\\
%     \midrule
%     \O & 1 in 1,000& For Swedish names\\
%     $\pi$ & 1 in 5& Common in math\\
%     \$ & 4 in 5 & Used in business\\
%     $\Psi^2_1$ & 1 in 40,000& Unexplained usage\\
%   \bottomrule
% \end{tabular}
% \end{table}

% To set a wider table, which takes up the whole width of the page's
% live area, use the environment \textbf{table*} to enclose the table's
% contents and the table caption.  As with a single-column table, this
% wide table will ``float'' to a location deemed more
% desirable. Immediately following this sentence is the point at which
% Table~\ref{tab:commands} is included in the input file; again, it is
% instructive to compare the placement of the table here with the table
% in the printed output of this document.

% \begin{table*}
%   \caption{Some Typical Commands}
%   \label{tab:commands}
%   \begin{tabular}{ccl}
%     \toprule
%     Command &A Number & Comments\\
%     \midrule
%     \texttt{{\char'134}author} & 100& Author \\
%     \texttt{{\char'134}table}& 300 & For tables\\
%     \texttt{{\char'134}table*}& 400& For wider tables\\
%     \bottomrule
%   \end{tabular}
% \end{table*}

% \subsection*{Math Equations}
% You may want to display math equations in three distinct styles:
% inline, numbered or non-numbered display.  Each of the three are
% discussed in the next sections.

% \subsubsection*{Inline (In-text) Equations}
% A formula that appears in the running text is called an inline or
% in-text formula.  It is produced by the \textbf{math} environment,
% which can be invoked with the usual
% \texttt{{\char'134}begin\,\ldots{\char'134}end} construction or with
% the short form \texttt{\$\,\ldots\$}. You can use any of the symbols
% and structures, from $\alpha$ to $\omega$, available in
% \LaTeX~\cite{Lamport:LaTeX}; this section will simply show a few
% examples of in-text equations in context. Notice how this equation:
% \begin{math}
%   \lim_{n\rightarrow \infty}x=0
% \end{math},
% set here in in-line math style, looks slightly different when
% set in display style.  (See next section).

% \subsubsection*{Display Equations}
% A numbered display equation---one set off by vertical space from the
% text and centered horizontally---is produced by the \textbf{equation}
% environment. An unnumbered display equation is produced by the
% \textbf{displaymath} environment.

% Again, in either environment, you can use any of the symbols and
% structures available in \LaTeX\@; this section will just give a couple
% of examples of display equations in context.  First, consider the
% equation, shown as an inline equation above:
% \begin{equation}
%   \lim_{n\rightarrow \infty}x=0
% \end{equation}
% Notice how it is formatted somewhat differently in
% the \textbf{displaymath}
% environment.  Now, we'll enter an unnumbered equation:
% \begin{displaymath}
%   \sum_{i=0}^{\infty} x + 1
% \end{displaymath}
% and follow it with another numbered equation:
% \begin{equation}
%   \sum_{i=0}^{\infty}x_i=\int_{0}^{\pi+2} f
% \end{equation}
% just to demonstrate \LaTeX's able handling of numbering.

% \subsection*{Figures}

% The ``\verb|figure|'' environment should be used for figures. One or
% more images can be placed within a figure. If your figure contains
% third-party material, you must clearly identify it as such, as shown
% in the example below.
% \begin{figure}[h]
%   \centering
%   \includegraphics[width=\linewidth]{sample-franklin}
%   \caption{1907 Franklin Model D roadster. Photograph by Harris \&
%     Ewing, Inc. [Public domain], via Wikimedia
%     Commons. (\url{https://goo.gl/VLCRBB}).}
%   \Description{The 1907 Franklin Model D roadster.}
% \end{figure}

% Your figures should contain a caption which describes the figure to
% the reader.
% Figure captions are placed {\itshape below} the figure.



% \subsection*{Citations and Bibliographies}

% The use of \emph{BibTeX} for the preparation and formatting of one's
% references is strongly recommended. Authors' names should be complete
% --- use full first names (``Donald E. Knuth'') not initials
% (``D. E. Knuth'') --- and the salient identifying features of a
% reference should be included: title, year, volume, number, pages,
% article DOI, etc.

% The bibliography is included in your source document with these two
% commands, placed just before the \verb|\end{document}| command:
% \begin{verbatim}
%   \bibliographystyle{ACM-Reference-Format}
%   \bibliography{bibfile}
% \end{verbatim}
% where ``\verb|bibfile|'' is the name, without the ``\verb|.bib|''
% suffix, of the \emph{BibTeX} file.

%   Some examples.  A paginated journal article \cite{Abril07}, an
%   enumerated journal article \cite{Cohen07}, a reference to an entire
%   issue \cite{JCohen96}, a monograph (whole book) \cite{Kosiur01}, a
%   monograph/whole book in a series (see 2a in spec. document)
%   \cite{Harel79}, a divisible-book such as an anthology or compilation
%   \cite{Editor00} followed by the same example, however we only output
%   the series if the volume number is given \cite{Editor00a} (so
%   Editor00a's series should NOT be present since it has no vol. no.),
%   a chapter in a divisible book \cite{Spector90}, a chapter in a
%   divisible book in a series \cite{Douglass98}, a multi-volume work as
%   book \cite{Knuth97}, an article in a proceedings (of a conference,
%   symposium, workshop for example) (paginated proceedings article)
%   \cite{Andler79}, a proceedings article with all possible elements
%   \cite{Smith10}, an example of an enumerated proceedings article
%   \cite{VanGundy07}, an informally published work \cite{Harel78}, a
%   doctoral dissertation \cite{Clarkson85}, a master's thesis:
%   \cite{anisi03}, an online document / world wide web resource
%   \cite{Thornburg01, Ablamowicz07, Poker06}, a video game (Case 1)
%   \cite{Obama08} and (Case 2) \cite{Novak03} and \cite{Lee05} and
%   (Case 3) a patent \cite{JoeScientist001}, work accepted for
%   publication \cite{rous08}, 'YYYYb'-test for prolific author
%   \cite{SaeediMEJ10} and \cite{SaeediJETC10}. Other cites might
%   contain 'duplicate' DOI and URLs (some SIAM articles)
%   \cite{Kirschmer:2010:AEI:1958016.1958018}. Boris / Barbara Beeton:
%   multi-volume works as books \cite{MR781536} and \cite{MR781537}. A
%   couple of citations with DOIs:
%   \cite{2004:ITE:1009386.1010128,Kirschmer:2010:AEI:1958016.1958018}. Online
%   citations: \cite{TUGInstmem, Thornburg01, CTANacmart}.


% \subsection*{Appendices}

% If your work needs an appendix, add it before the
% ``\verb|\end{document}|'' command at the conclusion of your source
% document.

% Start the appendix with the ``\verb|appendix|'' command:
% \begin{verbatim}
%   \appendix
% \end{verbatim}
% and note that in the appendix, sections are lettered, not
% numbered. This document has two appendices, demonstrating the section
% and subsection identification method.


%%
%% The next two lines define the bibliography style to be used, and
%% the bibliography file.
\bibliographystyle{ACM-Reference-Format}
\bibliography{report}

%%
%% If your work has an appendix, this is the place to put it.
\appendix

% \section{Research Methods (remove if not used)}

% \subsection{Part One}

% Lorem ipsum dolor sit amet, consectetur adipiscing elit. Morbi
% malesuada, quam in pulvinar varius, metus nunc fermentum urna, id
% sollicitudin purus odio sit amet enim. Aliquam ullamcorper eu ipsum
% vel mollis. Curabitur quis dictum nisl. Phasellus vel semper risus, et
% lacinia dolor. Integer ultricies commodo sem nec semper.

% \subsection{Part Two}

% Etiam commodo feugiat nisl pulvinar pellentesque. Etiam auctor sodales
% ligula, non varius nibh pulvinar semper. Suspendisse nec lectus non
% ipsum convallis congue hendrerit vitae sapien. Donec at laoreet
% eros. Vivamus non purus placerat, scelerisque diam eu, cursus
% ante. Etiam aliquam tortor auctor efficitur mattis.

% \section{Online Resources}

% Nam id fermentum dui. Suspendisse sagittis tortor a nulla mollis, in
% pulvinar ex pretium. Sed interdum orci quis metus euismod, et sagittis
% enim maximus. Vestibulum gravida massa ut felis suscipit
% congue. Quisque mattis elit a risus ultrices commodo venenatis eget
% dui. Etiam sagittis eleifend elementum.

% Nam interdum magna at lectus dignissim, ac dignissim lorem
% rhoncus. Maecenas eu arcu ac neque placerat aliquam. Nunc pulvinar
% massa et mattis lacinia.

\end{document}